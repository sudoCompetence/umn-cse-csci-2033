%	BEGIN: PACKAGES AND OTHER DOCUMENT CONFIGURATIONS
%----------------------------------------------------------------------------------------
\documentclass[
12pt, % Default font size, select one of 10pt, 11pt or 12pt
a4paper, % Paper size, use either 'a4paper' for A4 size or 'letterpaper' for US letter size
fleqn,   % Left align equations
% Uncomment for oneside mode, this doesn't start new chapters and parts on odd pages
% (adding an empty page if required), this mode is more suitable if the book is to be read on a screen instead of printed
oneside, 
]{master}
% \documentclass[12pt, letterpaper]{book}

% Book information for PDF metadata, remove/comment this block if not required 
\hypersetup{
    pdftitle={Title}, % Title field
    pdfauthor={Author}, % Author field
    pdfsubject={Subject}, % Subject field
    pdfkeywords={Keyword1, Keyword2, ...}, % Keywords
    pdfcreator={LaTeX}, % Content creator field
}
    
% Definitions
\definecolor{ocre}{RGB}{128, 0, 0} % Define the color used for highlighting throughout the book

% Opening background
\chapterimage{chapter-cover-one.jpg} % Chapter heading image
\chapterspaceabove{6.5cm} % Default whitespace from the top of the page to the chapter title on chapter pages
\chapterspacebelow{6.75cm} % Default amount of vertical whitespace from the top margin to the start of the text on chapter pages

% Bibliography
\newcites{Math}{Math Readings}
\newcites{Phys}{Physics Readings}
\lstset{
    basicstyle=\footnotesize\ttfamily,
    literate={~} {$\sim$}{1},
}

%	BEGIN: Document
%----------------------------------------------------------------------------------------
\begin{document}

%	BEGIN: TITLE PAGE
%----------------------------------------------------------------------------------------
\titlepage % Output the title page
{\includegraphics[width=\paperwidth]{umn-cse-cover.png}} % Code to output the background image, which should be the same dimensions as the paper to fill the page entirely; leave empty for no background image
{ % Title(s) and author(s)
    \centering\sffamily % Font styling
}

{ % Title(s) and author(s)
    \centering\sffamily % Font styling

    \vspace{1cm}
    \includegraphics[width=\textwidth]{umn-centered-digital.png}

    % {\large University of Minnesota \par}
    \vspace{1cm}
    {\LARGE College of Science and Engineering}

    \vfill
    \vspace{2cm}
    {\LARGE\ Subject Matter\par} % Author name
    \vspace{0.5cm}
    {\large Tyson M. Koch\par} % author 

    \vfill
    \includegraphics[width=.5\textwidth]{umn-logo.png}\hfill

    \vspace{1cm}
    {\small
        % Author: Tyson M. Koch, A.S. Natural Science \\
        Advisor: Professor Name, Ph.D.\\
        % Assistant: Teacher Assistant Name
    }

    \vspace{1cm}
    {\normalsize \today \par}
    % {\normalsize November 6, 2025 \par}
}

%	BEGIN: COPYRIGHT PAGE
%----------------------------------------------------------------------------------------
\newpage
\thispagestyle{empty} % Suppress headers and footers on this page

\includegraphics[width=.5\textwidth]{goldy-facing.png}\hfill

~\vfill % Push the text down to the bottom of the page

\noindent Copyright \copyright\ 2025 Tyson M. Koch\\ % Copyright notice

\noindent \textsc{Published by \\ The University of Minnesota - College of Science and Engineering}\\ % Publisher

\noindent \textsc{\href{https://www.latextemplates.com/template/legrand-orange-book}}\\ % URL

\noindent Licensed under the Creative Commons Attribution-NonCommercial 4.0 License (the ``License''). You may not use this file except in compliance with the License. You may obtain a copy of the License at \url{https://creativecommons.org/licenses/by-nc-sa/4.0}. Unless required by applicable law or agreed to in writing, software distributed under the License is distributed on an \textsc{``as is'' basis, without warranties or conditions of any kind}, either express or implied. See the License for the specific language governing permissions and limitations under the License.\\ % License information, replace this with your own license (if any)

\noindent \textit{First printed, January 2025 by Tyson M. Koch} % Printing/edition date

%	BEGIN: TABLE OF CONTENTS
%----------------------------------------------------------------------------------------
\pagestyle{empty} % Disable headers and footers for the following pages
\tableofcontents % Output the table of contents
\listoffigures % Output the list of figures, comment or remove this command if not required
\listoftables % Output the list of tables, comment or remove this command if not required

\newpage
\begin{figure}[b] % Floating figure, [b] tells LaTeX to place it at the bottom of the next available page
    \centering % Horizontally center the figure on the page
    \includegraphics[width=\textwidth]{umn-horizontal-digital.png}
    \includegraphics[width=\textwidth]{goldy-running.png}\par
    % \label{fig:umn-logo} % Unique label used for referencing the figure in-text
\end{figure}

\pagestyle{fancy} % Enable default headers and footers again

% \cleardoublepage % Start the following content on a new page


% ---------------------------------------------------------------------------------------
% WORKING: Edit here
% ---------------------------------------------------------------------------------------


% WARNING: STOP!
% ---------------------------------------------------------------------------------------
% WARNING: Do Not Not Edit Below!
% ---------------------------------------------------------------------------------------
% WARNING: STOP!
% ---------------------------------------------------------------------------------------


% Part: Part I: Part Name
% ---------------------------------------------------------------------------------------
\part{Part I: Part Name}

% Chapter: Chapter Name
% ---------------------------------------------------------------------------------------
\chapterimage{chapter-cover-two.jpg} % Chapter heading image
\chapterspaceabove{6.75cm} % Whitespace from the top of the page to the chapter title on chapter pages
\chapterspacebelow{7.25cm} % Amount of vertical whitespace from the top margin to the start of the text on chapter pages
\chapter{Introductions}\index{Introductions}

% Section: Section Name
% ---------------------------------------------------------------------------------------
\section{Welcome}\index{Introductions!Welcome}
This is a simple paragraph in LaTeX.

% Part: Part I: Part Name
% ---------------------------------------------------------------------------------------
\part{Part II: Examples}

% Chapter: Chapter Name
% ---------------------------------------------------------------------------------------
\chapterimage{chapter-cover-two.jpg} % Chapter heading image
\chapterspaceabove{6.75cm} % Whitespace from the top of the page to the chapter title on chapter pages
\chapterspacebelow{7.25cm} % Amount of vertical whitespace from the top margin to the start of the text on chapter pages
\chapter{Introductions}\index{Introductions}

% Section: Introductions
% ---------------------------------------------------------------------------------------
\section{Basics}\index{Introductions!Basics}
This is a simple paragraph in LaTeX.

This is a new paragraph. Create a new paragraph by leaving a blank line.

This line ends here.\\
This line starts after a manual line break.

You can make text \textbf{bold}, \textit{italic}, or \textbf{\textit{both}}. \\
\noindent
You can also use \emph{emphasis}, which is usually italicized.

% Subsection: Lists
% ---------------------------------------------------------------------------------------
\subsection{Enumerated List}\index{Basics!Enumerate}
\begin{enumerate}
    \item First numbered item
        \begin{enumerate}
            \item First indented numbered item
            \item Second indented numbered item
                \begin{enumerate}
                    \item First second-level indented numbered item
                \end{enumerate}
        \end{enumerate}
    \item Second numbered item
    \item Third numbered item
\end{enumerate}

% Subsection: Lists
% ---------------------------------------------------------------------------------------
\subsection{Itemize List}\index{Basics!Itemize}
\begin{itemize}
    \item First bullet point item
        \begin{itemize}
            \item First indented bullet point item
            \item Second indented bullet point item
                \begin{itemize}
                    \item First second-level indented bullet point item
                \end{itemize}
        \end{itemize}
    \item Second bullet point item
    \item Third bullet point item
\end{itemize}

\newpage
% Section: Matricies and Tables
% ---------------------------------------------------------------------------------------
\section{Matrices and Tables}\index{Introductions!Tables}

% Subsection: Matrix
% ---------------------------------------------------------------------------------------
\subsection{Matrix}\index{Tables!Matrix}
\[
    A =
    \begin{bmatrix}
    1 & 2 & 3 \\
    4 & 5 & 6
    \end{bmatrix}
\]

% Subsection: Matrix
% ---------------------------------------------------------------------------------------
\subsection{Augmented Matrix}\index{Tables!Matrix}
\[
    M_{1} =
    \left[
        \begin{array}{ccc|c}
        1 & 0 & -2 & 3 \\
        0 & 1 &  4 & 6 \\
        2 & -1 & 0 & -3
        \end{array}
    \right]
\]

% Subsection: Tables
% ---------------------------------------------------------------------------------------
\subsection{Tables}\index{Tables!Matrix}

\begin{table}[h]
    \centering
    \renewcommand{\arraystretch}{1.25} % spacing
        \begin{tabular}{||c c c c c c||}
            \hline
            $x$     & 0    & 1    & 2    & 3    & 4    \\
            \hline
            $f(x)$  & 0.41 & 0.37 & 0.16 & 0.05 & 0.01 \\
            \hline
        \end{tabular}

    \caption{Example Table I}
    \label{tab:exampletableone}
\end{table}
Referencing \autoref{tab:exampletableone} in-text using its label.

% Section: Figures 
% ---------------------------------------------------------------------------------------
\section{Figures}\index{Introductions!Figures}
\begin{figure}[H] % Use [H] to suppress floating and place the figure/table exactly where it is specified in the text
    \centering % Horizontally center the figure on the page
    \includegraphics[width=0.35\textwidth]{background.pdf} % Include the figure image
    \caption{Example Figure.}
    \label{fig:example} % Unique label used for referencing the figure in-text
\end{figure}
Referencing \autoref{fig:example} in-text using its label. \\

% \begin{figure}[b] % Floating figure, [b] tells LaTeX to place it at the bottom of the next available page
%     \centering % Horizontally center the figure on the page
%     \includegraphics[width=\textwidth]{background.pdf} % Include the figure image
%     \caption{Floating figure.}
%     \label{fig:floating} % Unique label used for referencing the figure in-text
% \end{figure}

\newpage
% Section: Citations 
% ---------------------------------------------------------------------------------------
\section{Citations}\index{Introductions!Citations}
According to Walpole (2019), "students often had difficulty using \textbf{APA style}, especially when it was their first time" (p. 1).

\noindent
This statement requies a \textbf{citation} \cite{rosen-2019} found "students often had difficulty using APA style".

This statement involkes a \textbf{footnote}, Walpole (2019) found "students often had difficulty using APA style".
\footnote{Ronald Walpole, “Scientific Data,” essay, in Introduction to Statistics and Probability
for Engineers and Scientists, ed. Raymond Myers, 9th ed. (Boston, MA: Pearson, 2019), p.1}

% Section: Links
% ---------------------------------------------------------------------------------------
\section{Link}\index{Introductions!Links}

This is a URL link: \href{https://www.latextemplates.com}{LaTeX Templates}. This is an email link: \href{mailto:example@example.com}{example@example.com}. This is a monospaced URL link: \url{https://www.LaTeXTemplates.com}.

% Section: Code
% ---------------------------------------------------------------------------------------
\section{Code}\index{Introductions!Code}
\begin{lstlisting}[language=C]
/* prompt user for input */
printf("prompt: ");

/* scanf standard input stream */
scanf("%modifier", address);
\end{lstlisting}

\lstinputlisting[label=snippet, caption=Coding Snippet, language=C]{source/struct.c}

% Section: Basic Mathematics
% ---------------------------------------------------------------------------------------
\section{Basic Mathematics}\index{Introductions!Mathematics}
Displayed equations appear centered on their own line: \\
Inline math appears within text, like $a^2 + b^2 = c^2$.

% Subsection: Fractions
% ---------------------------------------------------------------------------------------
\subsection{Fractions}\index{Mathematics!Fractions}
\[
\frac{x + 1}{x - 3}
\]

% Subsection: Subscripts & Superscripts 
% ---------------------------------------------------------------------------------------
\subsection{Superscripts \& Subscripts}\index{Mathematics!Exponets}

% Subsubsection: Superscripts 
% ---------------------------------------------------------------------------------------
\subsubsection{Superscripts}\index{Exponets!Superscripts}
$x^2$, $a^{n+1}$, $e^{i\pi} + 1 = 0$.

% Subsubsection: Subscripts
% ---------------------------------------------------------------------------------------
\subsubsection{Subscripts}\index{Exponets!Subscripts}
$x_1$, $a_{n}$, $A_{ij}$.

\newpage
% Section: Propositions 
% ---------------------------------------------------------------------------------------
\section{Propositions}\index{Introductions!Propositions}

\begin{proposition} % Specify a name/title in square brackets, or leave them out for no title
    Let $f,g\in L^2(G)$; if $\forall \varphi\in\mathcal{D}(G)$, $(f,\varphi)_0=(g,\varphi)_0$ then $f = g$. 
\end{proposition}

\begin{proposition}[Proposition name] % Specify a name/title in square brackets, or leave them out for no title
    It has the properties:
    \begin{align}
        & \big| ||\mathbf{x}|| - ||\mathbf{y}|| \big|\leq || \mathbf{x}- \mathbf{y}||\\
        &  ||\sum_{i=1}^n\mathbf{x}_i||\leq \sum_{i=1}^n||\mathbf{x}_i||\quad\text{where $n$ is a finite integer}
    \end{align}
\end{proposition}

% Subsection: Several Equations
%------------------------------------------------
\subsection{Several equations}\index{Propositions!SeveralEquations}

\begin{theorem}[Name of the theorem] % Specify a name/title in square brackets, or leave them out for no title
    In $E=\mathbb{R}^n$ all norms are equivalent. It has the properties:
    \begin{align}
        & \big| ||\mathbf{x}|| - ||\mathbf{y}|| \big|\leq || \mathbf{x}- \mathbf{y}||\\
        &  ||\sum_{i=1}^n\mathbf{x}_i||\leq \sum_{i=1}^n||\mathbf{x}_i||\quad\text{where $n$ is a finite integer}
    \end{align}
\end{theorem}

% Section: Theorems
%------------------------------------------------
\section{Theorems}\index{Introductions!Theorems}

\begin{theorem} % Specify a name/title in square brackets, or leave them out for no title
    A set $\mathcal{D}(G)$ in dense in $L^2(G)$, $|\cdot|_0$. 
\end{theorem}

% Section: Definitions Examples 
%------------------------------------------------
\section{Definitions}\index{Introductions!Definitions}

\begin{definition}[Definition name] % Specify a name/title in square brackets, or leave them out for no title
    Given a vector space $E$, a norm on $E$ is an application, denoted $||\cdot||$, $E$ in $\mathbb{R}^+=[0,+\infty[$ such that:
    \begin{align}
        & ||\mathbf{x}||=0\ \Rightarrow\ \mathbf{x}=\mathbf{0}\\
        & ||\lambda \mathbf{x}||=|\lambda|\cdot ||\mathbf{x}||\\
        & ||\mathbf{x}+\mathbf{y}||\leq ||\mathbf{x}||+||\mathbf{y}||
    \end{align}
\end{definition}

% Section: Examples 
%------------------------------------------------
\section{Examples}\index{Introductions!Examples}

\begin{example} % Specify a name/title in square brackets, or leave them out for no title
    Let $G=\{x\in\mathbb{R}^2:|x|<3\}$ and denoted by: $x^0=(1,1)$:
    \begin{equation}
        f(x)=\left\{\begin{aligned} & \mathrm{e}^{|x|} & & \text{si $|x-x^0|\leq 1/2$}\\
        & 0 & & \text{si $|x-x^0|> 1/2$}\end{aligned}\right.
    \end{equation}
    The function $f$ has bounded support, we can take $A=\{x\in\mathbb{R}^2:|x-x^0|\leq 1/2+\epsilon\}$ for all $\epsilon\in\mathopen{]}0\,;5/2-\sqrt{2}\mathclose{[}$.
\end{example}

% Subsection: Text Examples 
%------------------------------------------------
\section{Text Example}\index{Examples!Text}

\begin{example}[Example name] % Specify a name/title in square brackets, or leave them out for no title
    Aliquam arcu turpis, ultrices sed luctus ac, vehicula id metus. Morbi eu feugiat velit, et tempus augue.
    Donec cursus maximus luctus. Vivamus lobortis eros et massa porta porttitor.
\end{example}

\newpage
% Section: Exercises Examples 
%------------------------------------------------
\section{Exercises}\index{Introductions!Exercises}

\begin{exercise} % Specify a name/title in square brackets, or leave them out for no title
    This is a good place to ask a question to test learning progress or further cement ideas into students' minds.
\end{exercise}

% Section: Equation Problems 
%------------------------------------------------
\section{Problems}\index{Introductions!Problems}

\begin{problem} % Specify a name/title in square brackets, or leave them out for no title
    What is the average airspeed velocity of an unladen swallow?
\end{problem}


% Section: Notations
%------------------------------------------------
\section{Notations}\index{Introductions!Notations}
\begin{notation} % Specify a name/title in square brackets, or leave them out for no title
    Given an open subset $G$ of $\mathbb{R}^n$, the set of functions $\varphi$ are:
    \begin{enumerate}
        \item Bounded support $G$;
        \item Infinitely differentiable;
    \end{enumerate}
    a vector space is denoted by $\mathcal{D}(G)$. 
\end{notation}

% Section: Corollaries 
%------------------------------------------------
\section{Corollaries}\index{Introductions!Corollaries}
\begin{corollary}[Corollary name] % Specify a name/title in square brackets, or leave them out for no title
    The concepts presented here are now in conventional employment in mathematics. Vector spaces are taken over the field $\mathbb{K}=\mathbb{R}$, however, established properties are easily extended to $\mathbb{K}=\mathbb{C}$.
\end{corollary}

% Section: Remarks
%------------------------------------------------
\section{Remarks}\index{Introductions!Remarks}
\begin{remark}
    The concepts presented here are now in conventional employment in mathematics. Vector spaces are taken over the field $\mathbb{K}=\mathbb{R}$, however, established properties are easily extended to $\mathbb{K}=\mathbb{C}$.
\end{remark}

% Section: Vocabulary 
%------------------------------------------------
\section{Vocabulary}\index{Introductions!Vocabulary}

\begin{vocabulary}[Word] % Specify a name/title in square brackets, or leave them out for no title
    Definition of word.
\end{vocabulary}

\newpage
% Section: Calculus 
% ---------------------------------------------------------------------------------------
\section{Calculus}\index{Mathematics!Calculus}

% Subsection: Cases 
% ---------------------------------------------------------------------------------------
\subsection{Cases}\index{Calculus!Cases}
\textbf{Cases:} \\
\begin{flalign*}
    \tilde{x} =
    \begin{cases}
        x_{(n-1)/2} & \text{if n is odd} \\
        \frac{1}{2}(x_{n/2} + x_{n/2+1}) & \text{if n is even} \\
    \end{cases}
\end{flalign*} \\

% Subsection: Limits & Summations 
% ---------------------------------------------------------------------------------------
\subsection{Limits \& Summations}\index{Calculus!Summations}
\noindent
\textbf{Summations:} \\
\begin{flalign*}
    \overline{x} = \sum_{i=1}^{n} \frac{x_i}{n} = \frac{x_1 + x_2 + \dots + x_n}{n}
\end{flalign*} \\

\noindent
\textbf{Limits:} \\
\[
\lim_{x \to 0} \frac{\sin x}{x} = 1
\] \\

% Subsection: Derivatives 
% ---------------------------------------------------------------------------------------
\subsection{Derivatives}\index{Calculus!Derivatives}
\textbf{Derivative}
\[
\frac{d}{dx} x^2 = 2x
\] \\

\noindent
\textbf{Higher-order derivative:}
\[
\frac{d^2}{dx^2} x^3 = 6x
\] \\

% Subsection: Integrals 
% ---------------------------------------------------------------------------------------
\subsection{Integrals}\index{Calculus!Integrals}
\textbf{Indefinite:}
\[
\int x^2 \, dx = \frac{x^3}{3} + C
\] \\

\noindent
\textbf{Definite:}
\[
\int_{0}^{1} x^2 \, dx = \frac{1}{3}
\]

% Part: Part: End Matter
% ---------------------------------------------------------------------------------------
\part{Part: End Matter}

%----------------------------------------------------------------------------------------
\stopcontents[part] % Manually stop the 'part' table of contents here so the previous

%----------------------------------------------------------------------------------------
%	INDEX
%----------------------------------------------------------------------------------------

\cleardoublepage % Make sure the index starts on an odd (right side) page
\phantomsection
\addcontentsline{toc}{chapter}{\textcolor{ocre}{Index}} % Add an Index heading to the table of contents
\printindex % Output the index

%----------------------------------------------------------------------------------------
%	BIBLIOGRAPHY
%----------------------------------------------------------------------------------------

% \bibliographystyle{unsrt}
\bibliographystyle{apalike}
\bibliography{references}
\markboth{\sffamily\normalsize\bfseries Bibliography}{\sffamily\normalsize\bfseries Bibliography} % Set the page headers to display a Bibliography chapter name
\addcontentsline{toc}{chapter}{\textcolor{ocre}{Bibliography}} % Add a Bibliography heading to the table of contents

%----------------------------------------------------------------------------------------
%	APPENDICES
%----------------------------------------------------------------------------------------

\chapterimage{chapter-cover-two.jpg} % Chapter heading image
\chapterspaceabove{6.75cm} % Whitespace from the top of the page to the chapter title on chapter pages
\chapterspacebelow{7.25cm} % Amount of vertical whitespace from the top margin to the start of the text on chapter pages

\begin{appendices}
    \renewcommand{\chaptername}{Appendix} % Change the chapter name to Appendix, i.e. "Appendix A: Title", instead of "Chapter A: Title" in the headers
    \chapter{Appendix}
    \section{Appendix Section Title}
    Lorem ipsum dolor sit amet, consectetur adipiscing elit. Aliquam auctor mi risus, quis tempor libero hendrerit at.
    Duis hendrerit placerat quam et semper. Nam ultricies metus vehicula arcu viverra, vel ullamcorper justo elementum.
    Pellentesque vel mi ac lectus cursus posuere et nec ex. Fusce quis mauris egestas lacus commodo venenatis.
    Ut at arcu lectus. Donec et urna nunc. Morbi eu nisl cursus sapien eleifend tincidunt quis quis est. Donec ut orci ex.
    Praesent ligula enim, ullamcorper non lorem a, ultrices volutpat dolor. Nullam at imperdiet urna. 
    Pellentesque nec velit eget est euismod pretium.
\end{appendices}

%----------------------------------------------------------------------------------------
\end{document}
%----------------------------------------------------------------------------------------

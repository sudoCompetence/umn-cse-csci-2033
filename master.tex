%	BEGIN: PACKAGES AND OTHER DOCUMENT CONFIGURATIONS
%----------------------------------------------------------------------------------------
\documentclass[
12pt, % Default font size, select one of 10pt, 11pt or 12pt
a4paper, % Paper size, use either 'a4paper' for A4 size or 'letterpaper' for US letter size
fleqn,   % Left align equations
% Uncomment for oneside mode, this doesn't start new chapters and parts on odd pages
% (adding an empty page if required), this mode is more suitable if the book is to be read on a screen instead of printed
oneside, 
]{master}
% \documentclass[12pt, letterpaper]{book}

% Book information for PDF metadata, remove/comment this block if not required 
\hypersetup{
    pdftitle={Title}, % Title field
    pdfauthor={Author}, % Author field
    pdfsubject={Subject}, % Subject field
    pdfkeywords={Keyword1, Keyword2, ...}, % Keywords
    pdfcreator={LaTeX}, % Content creator field
}
    
% Definitions
\definecolor{ocre}{RGB}{128, 0, 0} % Define the color used for highlighting throughout the book

% Opening background
\chapterimage{chapter-cover-one.jpg} % Chapter heading image
\chapterspaceabove{6.5cm} % Default whitespace from the top of the page to the chapter title on chapter pages
\chapterspacebelow{6.75cm} % Default amount of vertical whitespace from the top margin to the start of the text on chapter pages

% Bibliography
\newcites{Math}{Math Readings}
\newcites{Phys}{Physics Readings}
\lstset{
    basicstyle=\footnotesize\ttfamily,
    literate={~} {$\sim$}{1},
}

%	BEGIN: Document
%----------------------------------------------------------------------------------------
\begin{document}

%	BEGIN: TITLE PAGE
%----------------------------------------------------------------------------------------
\titlepage % Output the title page
{\includegraphics[width=\paperwidth]{umn-cse-cover.png}} % Code to output the background image, which should be the same dimensions as the paper to fill the page entirely; leave empty for no background image
{ % Title(s) and author(s)
    \centering\sffamily % Font styling
}

{ % Title(s) and author(s)
    \centering\sffamily % Font styling

    \vspace{1cm}
    \includegraphics[width=\textwidth]{umn-centered-digital.png}

    % {\large University of Minnesota \par}
    \vspace{1cm}
    {\LARGE College of Science and Engineering}

    \vfill
    \vspace{2cm}
    {\LARGE\ Computational Linear Algebra\par} % Author name
    \vspace{0.5cm}
    {\large Tyson M. Koch\par} % author 

    \vfill
    \includegraphics[width=.5\textwidth]{umn-logo.png}\hfill

    \vspace{1cm}
    {\small
        Author: Tyson M. Koch, A.S. \\
        Advisor: Bernardo Bianco Prado M.S. \\
        % Assistant: Teacher Assistant Name
    }

    \vspace{1cm}
    {\normalsize \today \par}
    % {\normalsize November 6, 2025 \par}
}

%	BEGIN: COPYRIGHT PAGE
%----------------------------------------------------------------------------------------
\newpage
\thispagestyle{empty} % Suppress headers and footers on this page

\includegraphics[width=.5\textwidth]{goldy-facing.png}\hfill

~\vfill % Push the text down to the bottom of the page

\noindent Copyright \copyright\ 2025 Tyson M. Koch\\ % Copyright notice

\noindent \textsc{Published by \\ The University of Minnesota - College of Science and Engineering}\\ % Publisher

\noindent \textsc{\href{https://www.latextemplates.com/template/legrand-orange-book}}\\ % URL

\noindent Licensed under the Creative Commons Attribution-NonCommercial 4.0 License (the ``License''). You may not use this file except in compliance with the License. You may obtain a copy of the License at \url{https://creativecommons.org/licenses/by-nc-sa/4.0}. Unless required by applicable law or agreed to in writing, software distributed under the License is distributed on an \textsc{``as is'' basis, without warranties or conditions of any kind}, either express or implied. See the License for the specific language governing permissions and limitations under the License.\\ % License information, replace this with your own license (if any)

\noindent \textit{First printed, January 2025 by Tyson M. Koch} % Printing/edition date

%	BEGIN: TABLE OF CONTENTS
%----------------------------------------------------------------------------------------
\pagestyle{empty} % Disable headers and footers for the following pages
\tableofcontents % Output the table of contents
\listoffigures % Output the list of figures, comment or remove this command if not required
\listoftables % Output the list of tables, comment or remove this command if not required

\newpage
\begin{figure}[b] % Floating figure, [b] tells LaTeX to place it at the bottom of the next available page
    \centering % Horizontally center the figure on the page
    \includegraphics[width=\textwidth]{umn-horizontal-digital.png}
    \includegraphics[width=\textwidth]{goldy-running.png}\par
    % \label{fig:umn-logo} % Unique label used for referencing the figure in-text
\end{figure}

\pagestyle{fancy} % Enable default headers and footers again

% \cleardoublepage % Start the following content on a new page


% ---------------------------------------------------------------------------------------
% WORKING: Edit here
% ---------------------------------------------------------------------------------------

% Part: Introduction to Linear Algebra
% ---------------------------------------------------------------------------------------
\part{Introduction to Linear Algebra}

% Chapter: Linear Equations & Matricies
% ---------------------------------------------------------------------------------------
\chapterimage{chapter-cover-one.jpg} % Chapter heading image
\chapterspaceabove{6.75cm} % Whitespace from the top of the page to the chapter title on chapter pages
\chapterspacebelow{7.25cm} % Amount of vertical whitespace from the top margin to the start of the text on chapter pages
\chapter{Linear Equations and Matricies}\index{Matricies}

% Section: Linear Equaions
% ---------------------------------------------------------------------------------------
\section{Linear Equations}\index{Matricies!LinearEquations}
Linear algebra is a systematic analysis of linear equations whose applications
range from functional analysis to quantum mechanics. "In all of mathematics,
the concept of linearization is critical because linear problems are
very well understood and we can say a lot about them. For this reason we try to convert
many areas of mathematics to linear problems so that we can solve them." \cite{singh-2013}

\begin{definition}
    An \textbf{equation} is two equivalent mathematical statements.
\end{definition}

\begin{definition}
    \textbf{\textit{Linear Equation}} \\
    Equation such that each term is a \textbf{scaler} $s_n$,
    multiplied by a variables $v_n$,
    of \textbf{index (power) of 1 or 0}
    \begin{displaymath}
        (s_1)^1 (v_1)^1 + ... + (s_n)^1 (v_n)^1 = k
    \end{displaymath}
\end{definition}
\begin{remark}
    Note that if an equation contains an argument of trigonometric, exponential,
    logarithmic or hyperbolic functions then the equation is not linear.
\end{remark}

\begin{definition}
    \textbf{\textit{System of Linear Equation}} \\
    \textbf{Linear System} is a set of two or more linear equations such that:
    \begin{align*}
        (s_{11})^1 (v_1)^1 + (s_{12})^1 (v_1)^1 + ... + (s_{1n})^1 (v_n)^1 = k_1 \\
        (s_{21})^1 (v_2)^1 + (s_{22})^1 (v_2)^1 + ... + (s_{2n})^1 (v_n)^1 = k_2 \\
        (s_{31})^1 (v_3)^1 + (s_{32})^1 (v_3)^1 + ... + (s_{3n})^1 (v_n)^1 = k_3 \\
    \end{align*}
\end{definition}

\begin{corollary}
    Three operations of linear equations
    \begin{enumerate}
        \item Interchange \textbf{equations}
        \item Multiply \textit{\textbf{equations} by non-zero constant}
        \item Add or subtract \textit{one \textbf{equation} from another}
    \end{enumerate}
\end{corollary}

% Subsection: Solving Linear Equaions
% ---------------------------------------------------------------------------------------
\subsection{Solving Linear Equations}\index{LinearEquations!Solving}
\begin{exercise}
    \textbf{Solve Linear Equation}
    \begin{proposition}
        \textit{Let:}
        \begin{alignat}{3}
            3x + {} &&  y - {} &&  2z &= 4   \\
            5x - {} && 3y + {} && 10z &= 32  \\   
            7x + {} && 4y + {} && 16z &= 13
        \end{alignat}
    \end{proposition}

    \newpage\noindent Eliminate y
    \begin{alignat*}{3}
        (3)(3x + y  - 2z = 4) + {} && (5x - 3y + 10z = 32) &= (14x + 4z  = 4) \\
        (4)(3x + y  - 2z = 4) - {} && (7x + 4y + 16z = 13) &= ( 5x - 24z = 3)
    \end{alignat*}

    \noindent Eliminate z
    \begin{align*}
        & (6)(14x + 4z  = 4) + (5x - 24z = 3) = (89x = 267) \\
        & \therefore x = 3
    \end{align*}

    \noindent Back substitution 
    \begin{align*}
        & [5x - 24z = 3] \equiv [5(3) - 24z = 3] \mid (x=3) \\
        & \therefore z = \frac{1}{2} \\ \\
        & [3x + y - 2z = 4] \equiv [3(3) + y - 2(\frac{1}{2}) = 4] \mid (x=3, z=\frac{1}{2}) \\
        & \therefore y = -4
    \end{align*}
\end{exercise}

% Subsection: Intercepts & Graphing
% ---------------------------------------------------------------------------------------
\newpage
\subsection{Intercepts \& Graphing}\index{LinearEquaions!Intercepts}
\begin{exercise}
    \textbf{Intercepts \& Solution}
    \begin{proposition}
        \textit{Let:}
        \begin{alignat}{3}
            5x - {}&& 2y &= 5 \\
            3x - {}&& 2y &= 3
        \end{alignat}
    \end{proposition}

    \noindent Solve for $(x_1, y_1) \in (5x-2y=5) \mid (x=0, y=0)$
    \begin{alignat*}{2}
        x_1 = \frac{2(0)+5}{5} = 1 \\
        y_1 = \frac{5(0)-5}{2} = -\frac{5}{2}
    \end{alignat*}
    $\therefore (x_1, y_1) = (1, \frac{5}{2})$ \\

    \noindent Solve for $(x_2, y_2) \in (3x-2y=3) \mid (x=0, y=0)$
    \begin{alignat*}{2}
        x_2 = \frac{2(0)+3}{3} = 1 \\
        y_2 = \frac{3(0)-3}{2} = -\frac{3}{2}
    \end{alignat*}
    $\therefore (x_2, y_2) = (1, -\frac{3}{2})$

    \begin{figure}[H]
        \centering
        \includegraphics[width=0.35\textwidth]{math.png}
        \caption{X \& Y Intercepts.}
        % \label{fig:intercepts}
    \end{figure}
\end{exercise}

% Subsection: Matrix
% ---------------------------------------------------------------------------------------
\newpage
\subsection{Matrix}\index{Matrix!}
\begin{definition}
    \textbf{\textit{Matrix}}
    \begin{proposition}
        \textit{Let the sytem of linear equations be defined as:}
        \begin{align*}
            (s_{11})^1 (v_1)^1 + (s_{12})^1 (v_1)^1) + ... + (s_{1n})^1 (v_n)^1 = k_1 \\
            (s_{21})^1 (v_2)^1 + (s_{22})^1 (v_2)^1) + ... + (s_{2n})^1 (v_n)^1 = k_2 \\
            (s_{31})^1 (v_3)^1 + (s_{32})^1 (v_3)^1) + ... + (s_{3n})^1 (v_n)^1 = k_3
        \end{align*}
    \end{proposition}

    \begin{displaymath}
        M =
        \left[
            \begin{array}{ccc|c}
                s_{11} & s_{12} & s_{13} & k_1 \\
                s_{21} & s_{12} & s_{13} & k_2 \\
                .      & .      & .            \\
                s_{1n} & s_{1n} & s_{1n} & k_3
            \end{array}
        \right]
    \end{displaymath}
\end{definition}

\begin{corollary}
    Three operations of matricies
    \begin{enumerate}
        \item Interchange rows 
        \item Multiply \textit{\textbf{rows} by non-zero constant}
        \item Add or subtract \textit{one \textbf{row} from another}
    \end{enumerate}
\end{corollary}

% Section: Gaussian Elimination
% ---------------------------------------------------------------------------------------
\section{Gaussian Elimination}\index{Matrix!Gaussian}
"Johann Carl Friedrich Gauss, a prolific German mathematician,
is widely regarded as one of the three greatest mathematicians of all time,
the others being of course Archimedes and Sir Isaac Newton."

"Gauss went as a student to the world-renowned centre for mathematics – Göttingen.
Later in life, Gauss took up a post at Göttingen and published papers in number theory,
infinite series, algebra, astronomy and optics.
The unit of magnetic induction is (also) named after Gauss." \cite{singh-2013}

\begin{definition}[Reduced Row Form]
    \begin{displaymath}
        M =
        \left[
            \begin{array}{ccc|c}
                * & * & * & *\\
                0 & * & * & *\\
                0 & 0 & A & B
            \end{array}
        \right] \mid A = B
    \end{displaymath}
\end{definition}

\begin{definition}[Reduced Row Echelon Form]
    A matrix is in reduced row echelon form,
    normally abbreviated to \textbf{rref} if and only if: \\
    \begin{enumerate}
        \item \textbf{Zeroed Rows:} Rows with all zero must be at bottom of matrix.
        \item \textbf{Leading One:} Non-Zeroed Rows must have leading 1.
        \item \textbf{Columns Below Leading One:} Leading 1 only non-zero digit in column.
        \item \textbf{Consecutive Leading Ones:} The leading 1 from consecutive rows must go
            strictly from top-left to bottom-right on the matrix. \\
    \end{enumerate}
    \begin{displaymath}
        M =
        \left[
            \begin{array}{ccc|c}
                1 & 0 & 0 & * \\
                0 & 1 & 0 & * \\
                0 & 0 & 1 & *
            \end{array}
        \right]
    \end{displaymath}
\end{definition}

\newpage\begin{exercise}{Gaussian Elimination}
    \begin{proposition}
        \textit{Let:}
        \begin{alignat}{3}
            x  + {} && 3y + {} &&  2z &= 13  \\
            4x + {} && 4y - {} &&  3z &=  3  \\   
            5x + {} &&  y + {} &&  2z &= 13 
        \end{alignat}
    \end{proposition}

    \begin{displaymath}
        M =
        \left[
            \begin{array}{ccc|c}
                1 & 3 &  2 & 13 \\
                4 & 4 & -3 &  3 \\
                5 & 1 &  2 & 13
            \end{array}
        \right]
    \end{displaymath}

    \noindent Zero out first column
    \begin{displaymath}
        \begin{aligned}
            R_1 \\
            R_2^{\dagger} = R_2 - 4R_1 \\
            R_3^{\dagger} = R_3 - 5R_1 \\
        \end{aligned}
        \left(
            \begin{array}{ccc|c}
                1 & 3 & 2 & 13\\
                0 & -8 & -11 & -49\\
                0 & -14 & -8 & -52\\
            \end{array}
        \right)
    \end{displaymath}

    \noindent Zero out last row, second column
    \begin{displaymath}
        \begin{aligned}
            R_1 \\
            R_2^{\dagger} \\
            R_3^{\dagger\dagger} = R_3^{\dagger} - \frac{14}{8}R_2^{\dagger} \\
        \end{aligned}
        \left(
            \begin{array}{ccc|c}
                1 & 3 & 2 & 13\\
                0 & -8 & -11 & -49\\
                0 &  0 & \frac{45}{4} & \frac{135}{4} \\
            \end{array}
        \right)
    \end{displaymath}

    \noindent Solve for z
    \begin{displaymath}
        \begin{aligned}
            R_1 \\
            R_2^{\dagger} \\
            R_3^{\dagger\dagger\dagger} = \frac{4}{45}R_3^{\dagger\dagger}
        \end{aligned}
        \left(
            \begin{array}{ccc|c}
                1 & 3 & 2 & 13\\
                0 & -8 & -11 & -49\\
                0 &  0 & 1 & 3 \\
            \end{array}
        \right)
    \end{displaymath}

    \noindent Solve for $(y) \in (-8y-11z=-49) \mid z=3$
    \begin{align}
        (-8y - 11z = -49) \equiv [-8y-11(3)=-49] \equiv (-8y=-16)
    \end{align}
    $\therefore y = 2$ \\

    \noindent Solve for $(x) \in (x+3y+2z=13) \mid (y=2, z=3)$
    \begin{align}
        (x+3y+2z=13) \equiv [x+3(2)+2(3)=13] \equiv (x=1)
    \end{align}
    $\therefore (x=1, y=2, z=3)$
\end{exercise}

\newpage
\begin{exercise}{Reduced Row Echelon Form}
    \begin{proposition}
        \textit{Let:}
        \begin{displaymath}
            \begin{aligned}
                R_1 \\
                R_2 \\
                R_3
            \end{aligned}
            \left(
                \begin{array}{ccc|c}
                    1 & 3  &   2 &  13 \\
                    0 & -8 & -11 & -49 \\
                    0 &  0 &   1 &   3 \\
                \end{array}
            \right)
        \end{displaymath}
    \end{proposition}

    \noindent Resolve the second row, last column to 0
    \begin{displaymath}
        \begin{aligned}
            R_1 \\
            R_2^{\dagger} = R_2 - (11)(R_3) \\
            R_3
        \end{aligned}
        \left(
            \begin{array}{ccc|c}
                1 & 3  &   2 &  13 \\
                0 & -8 &   0 & -16 \\
                0 &  0 &   1 &   3 \\
            \end{array}
        \right)
    \end{displaymath}

    \noindent Resolve the second row, second column to 1
    \begin{displaymath}
        \begin{aligned}
            R_1 \\
            R_2^{\dagger\dagger} = (R_2^{\dagger})(-\frac{1}{8}) \\
            R_3
        \end{aligned}
        \left(
            \begin{array}{ccc|c}
                1  &  3  &   2  &  13 \\
                0  &  1  &   0  &   2 \\
                0  &  0  &   1  &   3 \\
            \end{array}
        \right)
    \end{displaymath}

    \noindent Resolve the second row, second column to 1
    \begin{displaymath}
        \begin{aligned}
            R_1^{\dagger} = R_1 - (2)(R_3) \\
            R_2^{\dagger\dagger} \\
            R_3
        \end{aligned}
        \left(
            \begin{array}{ccc|c}
                1  &  3  &   0  &   7 \\
                0  &  1  &   0  &   2 \\
                0  &  0  &   1  &   3 \\
            \end{array}
        \right)
    \end{displaymath}

    \begin{displaymath}
        \begin{aligned}
            R_1^{\dagger\dagger} = R_1^{\dagger} - (3)(R_2^{\dagger\dagger}) \\
            R_2^{\dagger\dagger} \\
            R_3
        \end{aligned}
        \left(
            \begin{array}{ccc|c}
                1  &  0  &   0  &   1 \\
                0  &  1  &   0  &   2 \\
                0  &  0  &   1  &   3 \\
            \end{array}
        \right)
    \end{displaymath}
    $\therefore (x=1, y=2, z=3)$
\end{exercise}

% Section: Vector Arithmetic 
% ---------------------------------------------------------------------------------------
\newpage
\section{Vector Arithmetic}\index{Introductions!Vectors}
With respect to vectors, physicists rely upon vectors to mathematically express the motion of an
object in terms of its direction, as well as the rate that it is travelling. Engineers will
use a vector to express the magnitude of a force and the direction in which it is acting.

\noindent\begin{definition}
    \textbf{Scaler} \\
    Numerical value that quantifies a qualitative phonomena.
    \textit{Denoted with numerical values}
\end{definition}

\noindent\begin{definition}
    \textbf{Vector} \
    Symbolic representation of a numerical quantities of \textit{magnitude \& direction.}
    \textit{Usually denoted with lowercase letter}
    \begin{displaymath}
        \vec{v}
    \end{displaymath}
\end{definition}

\noindent\begin{definition}
    \textbf{Vector Addition} \\
    Operation that combines two vectors of the same dimension by adding corresponding components, producing a new vector in the same space.
    \begin{displaymath}
        \vec{u} + \vec{v}
        =
        \begin{bmatrix}
            u_1 \\ u_2 \\ \vdots \\ u_n
        \end{bmatrix}
        +
        \begin{bmatrix}
            v_1 \\ v_2 \\ \vdots \\ v_n
        \end{bmatrix}
        =
        \begin{bmatrix}
            u_1+v_1 \\ u_2+v_2 \\ \vdots \\ u_n+v_n
        \end{bmatrix}
    \end{displaymath}
\end{definition}

\noindent\begin{definition}
    \textbf{Dot Product} \\
    Operation on two vectors of the same dimension that returns a scalar, computed as the sum of products of corresponding components.
    \begin{displaymath}
        \vec{u}\cdot\vec{v}
        =
        \begin{bmatrix}
            u_1 \\ u_2 \\ \vdots \\ u_n
        \end{bmatrix}
        \cdot
        \begin{bmatrix}
            v_1 \\ v_2 \\ \vdots \\ v_n
        \end{bmatrix}
        =
        \sum_{i=1}^{n} u_i v_i
        =
        u_1v_1 + u_2v_2 + \cdots + u_nv_n
    \end{displaymath}
\end{definition}

\noindent\begin{definition}
    \textbf{Linear Combination} \\
    Expression formed by scaling vectors and adding them together, producing a vector in the same space.
    \begin{displaymath}
        c_1\vec{v}_1 + c_2\vec{v}_2 + \cdots + c_k\vec{v}_k
    \end{displaymath}
    Where $c_1,c_2,\ldots,c_k$ are scalars and $\vec{v}_1,\vec{v}_2,\ldots,\vec{v}_k$ are vectors in $\mathbb{R}^n$.
    \begin{displaymath}
        c_1
        \begin{bmatrix}
            v_{11}\\ v_{12}\\ \vdots\\ v_{1n}
        \end{bmatrix}
        +
        c_2
        \begin{bmatrix}
            v_{21}\\ v_{22}\\ \vdots\\ v_{2n}
        \end{bmatrix}
        +
        \cdots
        +
        c_k
        \begin{bmatrix}
            v_{k1}\\ v_{k2}\\ \vdots\\ v_{kn}
        \end{bmatrix}
        =
        \begin{bmatrix}
            c_1v_{11}+c_2v_{21}+\cdots+c_kv_{k1}\\
            c_1v_{12}+c_2v_{22}+\cdots+c_kv_{k2}\\
            \vdots\\
            c_1v_{1n}+c_2v_{2n}+\cdots+c_kv_{kn}
        \end{bmatrix}
    \end{displaymath}
\end{definition}

\begin{figure}[H] % Use [H] to suppress floating and place the figure/table exactly where it is specified in the text
    \centering % Horizontally center the figure on the page
    \includegraphics[width=0.75\textwidth]{vectors/addition-one.png} % Include the figure image
    \caption{Vector Addition Example.}
    % \label{fig:vector-one} % Unique label used for referencing the figure in-text
\end{figure}
\begin{figure}[H] % Use [H] to suppress floating and place the figure/table exactly where it is specified in the text
    \centering % Horizontally center the figure on the page
    \includegraphics[width=0.75\textwidth]{vectors/addition-two.png} % Include the figure image
    \caption{Another Vector Addition Example.}
    % \label{fig:vector-one} % Unique label used for referencing the figure in-text
\end{figure}
\begin{figure}[H] % Use [H] to suppress floating and place the figure/table exactly where it is specified in the text
    \centering % Horizontally center the figure on the page
    \includegraphics[width=0.75\textwidth]{vectors/cordinates.png} % Include the figure image
    \caption{Vector Cordinates.}
    % \label{fig:vector-one} % Unique label used for referencing the figure in-text
\end{figure}
\begin{figure}[H] % Use [H] to suppress floating and place the figure/table exactly where it is specified in the text
    \centering % Horizontally center the figure on the page
    \includegraphics[width=0.5\textwidth]{vectors/r3.png} % Include the figure image
    \caption{Vector in $\mathbb{R}^3$}
    % \label{fig:vector-one} % Unique label used for referencing the figure in-text
\end{figure}

% Section: 
% ---------------------------------------------------------------------------------------
\section{Matrix Arithmetic}\index{Matricies!Arithmetic}
Computer graphics are essentially created using high-speed transformations. For example,
a computer animated sequence is based on modelling surfaces of connecting triangles. The
computer stores the vertices of the triangle in its memory and then certain operations such
as rotations, translations, reflections and enlargements are carried out by (transformation)
matrices.
In this context, we apply a (transformation) matrix in order to perform a function such
as rotation, reflection or translation, 

% Subsection: Matrix Addition
% ---------------------------------------------------------------------------------------
\subsection{Matrix Addition}\index{Arithmetic!Addition}
\noindent\begin{definition}
    \textbf{Matrix Addition} \\
    Operation that combines two matrices of the same dimension by adding corresponding entries, producing a new matrix of the same size.
    \begin{displaymath}
        A + B
        =
        \begin{bmatrix}
            a_{11} & a_{12} & \cdots & a_{1n} \\
            a_{21} & a_{22} & \cdots & a_{2n} \\
            \vdots & \vdots & \ddots & \vdots \\
            a_{m1} & a_{m2} & \cdots & a_{mn}
        \end{bmatrix}
        +
        \begin{bmatrix}
            b_{11} & b_{12} & \cdots & b_{1n} \\
            b_{21} & b_{22} & \cdots & b_{2n} \\
            \vdots & \vdots & \ddots & \vdots \\
            b_{m1} & b_{m2} & \cdots & b_{mn}
        \end{bmatrix}
        =
        \begin{bmatrix}
            a_{11}+b_{11} & a_{12}+b_{12} & \cdots & a_{1n}+b_{1n} \\
            a_{21}+b_{21} & a_{22}+b_{22} & \cdots & a_{2n}+b_{2n} \\
            \vdots & \vdots & \ddots & \vdots \\
            a_{m1}+b_{m1} & a_{m2}+b_{m2} & \cdots & a_{mn}+b_{mn}
        \end{bmatrix}
    \end{displaymath}
\end{definition}

% Subsection: Matrix-Vector Product
% ---------------------------------------------------------------------------------------
\subsection{Matrix-Vector Product}\index{Arithmetic!VectorMatrixProduct}
\noindent\begin{definition}
    \textbf{Matrix-Vector Product} \\
    A linear system can be written as a linear combination of column vectors, and equivalently in matrix form \(A\mathbf{x}=\mathbf{b}\).
    \begin{displaymath}
        \left[\begin{array}{c}
            2\\
            2
        \end{array}\right]x
        +
        \left[\begin{array}{c}
            2\\
            1
        \end{array}\right]y
        =
        \left[\begin{array}{c}
            3\\
            2.5
        \end{array}\right]
        \iff
        \begin{bmatrix}
            2 & 2\\
            2 & 1
        \end{bmatrix}
        \begin{bmatrix}
            x\\
            y
        \end{bmatrix}
        =
        \begin{bmatrix}
            3\\
            2.5
        \end{bmatrix}
    \end{displaymath}
\end{definition}

\newpage
\begin{exercise}{Matrix Form $Ax=b$}
    \begin{proposition}
        \textit{Let:}
        \begin{alignat}{3}
            x + {} && y + {} && 3z &= 5 \\
           -2x - {} && y + {} && 5z &= 6
        \end{alignat}
    \end{proposition}
    % Matrix form
    \begin{displaymath}
        A =
        \begin{bmatrix}
            1 & 1 & 3 \\
           -2 & -1 & 5
        \end{bmatrix},
        \quad
        \mathbf{x} =
        \begin{bmatrix}
            x \\ y \\ z
        \end{bmatrix},
        \quad
        \mathbf{b} =
        \begin{bmatrix}
            5 \\ 6
        \end{bmatrix}
    \end{displaymath}
    \begin{displaymath}
        A\mathbf{x}=\mathbf{b}
    \end{displaymath}
\end{exercise}

\begin{definition}
    \textbf{Matrix-Vector Multiplication} \\
    Operation that multiplies an \(m \times n\) matrix by an \(n \times 1\) vector, producing an \(m \times 1\) vector.
    \begin{displaymath}
        A\mathbf{x}
        =
        \begin{bmatrix}
            a_{11} & a_{12} & \cdots & a_{1n} \\
            a_{21} & a_{22} & \cdots & a_{2n} \\
            \vdots & \vdots & \ddots & \vdots \\
            a_{m1} & a_{m2} & \cdots & a_{mn}
        \end{bmatrix}
        \begin{bmatrix}
            x_1 \\
            x_2 \\
            \vdots \\
            x_n
        \end{bmatrix}
        =
        \begin{bmatrix}
            a_{11}x_1 + a_{12}x_2 + \cdots + a_{1n}x_n \\
            a_{21}x_1 + a_{22}x_2 + \cdots + a_{2n}x_n \\
            \vdots \\
            a_{m1}x_1 + a_{m2}x_2 + \cdots + a_{mn}x_n
        \end{bmatrix}
    \end{displaymath}
\end{definition}

% Subsection: Matrix-Matrix Product 
% ---------------------------------------------------------------------------------------
\subsection{Matrix-Matrix Product}\index{Arithmetic!MatrixMatrixProduct}
\begin{definition}
    \textbf{Matrix-Matrix Multiplication} \\
    Operation that multiplies an \(m \times n\) matrix by an \(n \times p\) matrix, producing an \(m \times p\) matrix. Each entry is the dot product of a row of the first matrix with a column of the second matrix.
    \begin{displaymath}
        AB
        =
        \begin{bmatrix}
            a_{11} & a_{12} & \cdots & a_{1n} \\
            a_{21} & a_{22} & \cdots & a_{2n} \\
            \vdots & \vdots & \ddots & \vdots \\
            a_{m1} & a_{m2} & \cdots & a_{mn}
        \end{bmatrix}
        \begin{bmatrix}
            b_{11} & b_{12} & \cdots & b_{1p} \\
            b_{21} & b_{22} & \cdots & b_{2p} \\
            \vdots & \vdots & \ddots & \vdots \\
            b_{n1} & b_{n2} & \cdots & b_{np}
        \end{bmatrix}
        =
        \begin{bmatrix}
            c_{11} & c_{12} & \cdots & c_{1p} \\
            c_{21} & c_{22} & \cdots & c_{2p} \\
            \vdots & \vdots & \ddots & \vdots \\
            c_{m1} & c_{m2} & \cdots & c_{mp}
        \end{bmatrix}
    \end{displaymath}
    where
    \begin{displaymath}
        c_{ij} = a_{i1}b_{1j} + a_{i2}b_{2j} + \cdots + a_{in}b_{nj}
        = \sum_{k=1}^{n} a_{ik}b_{kj}.
    \end{displaymath}
\end{definition}

\begin{example}
    \textbf{Matrix-Matrix Multiplication (2 \(\times\) 2 Form)} \\
    Multiply each row of the first matrix by each column of the second matrix (row-by-column dot products).
    \begin{displaymath}
        \begin{pmatrix}
            a & b \\
            c & d
        \end{pmatrix}
        \times
        \begin{pmatrix}
            e & f \\
            g & h
        \end{pmatrix}
        =
        \begin{pmatrix}
            ae+bg & af+bh \\
            ce+dg & cf+dh
        \end{pmatrix}
    \end{displaymath}
\end{example}

\begin{remark}
    In general, if $A$ is a $m \times r$ (m rows by r columns) matrix and
    $B$ is a $r \times n$ (r rows by n columns) matrix
    then the multiplication $AB$ results in a $m \times n$ matrix.
\end{remark}
\begin{remark}
    Matrix multiplication is only valid when the \textbf{\textit{inner}} dimensions
    of two matricies are equivalent. \\
    i.e. $(2\times4)\dot(4\times4)$ is \textbf{valid} \\
    i.e. $(4\times4)\dot(2\times4)$ is \textbf{invalid}
\end{remark}
\begin{figure}[H] % Use [H] to suppress floating and place the figure/table exactly where it is specified in the text
    \centering % Horizontally center the figure on the page
    \includegraphics[width=0.5\textwidth]{matrix/dimensions.png} % Include the figure image
    \caption{Resultant matrix multiplication dimensions}
    % \label{fig:vector-one} % Unique label used for referencing the figure in-text
\end{figure}

% Section: Matrix Algebra
% ---------------------------------------------------------------------------------------
\section{Matrix Algebra}\index{Matricies!MatrixAlgebra}
Now we shall examine the algebraic properties of vectors and matricies.
"Matrix powers are particularly useful in Markov chains – these are based on matrices whose
entries are probabilities. Many real life systems have an element of uncertainty and develop
over time and this can be explained through Markov chains.
A Markov chain is a sequence of random variables with the property that given the
present state, the future and past states are independent. For example, the game Monopoly
where the states are determined entirely by dice is a Markov chain. However, games like
poker are not a Markov chain because what is displayed depends on past moves.
See question 10 of Exercises 1.5 for a concrete example"" \cite{singh-2013}

Note that "Mathematical proof is a lot more powerful than a scientific proof because it is not subject
to experimental data, so once we have carried out a mathematical proof we know that our
result is absolutely correct and permanent." \cite{singh-2013}

\begin{definition}
    \textbf{Zero Matrix} \\
    A matrix in which every entry is \(0\). The zero matrix of size \(m \times n\) is denoted by \(0_{m\times n}\).
    \begin{displaymath}
        0_{m\times n}
        =
        \begin{bmatrix}
            0 & 0 & \cdots & 0 \\
            0 & 0 & \cdots & 0 \\
            \vdots & \vdots & \ddots & \vdots \\
            0 & 0 & \cdots & 0
        \end{bmatrix}
    \end{displaymath}
\end{definition}

\begin{definition}
    \textbf{Identity Matrix} \\
    A square matrix with \(1\)s on the main diagonal and \(0\)s elsewhere. The identity matrix of size \(n \times n\) is denoted by \(I_n\).
    \begin{displaymath}
        I_n
        =
        \begin{bmatrix}
            1 & 0 & \cdots & 0 \\
            0 & 1 & \cdots & 0 \\
            \vdots & \vdots & \ddots & \vdots \\
            0 & 0 & \cdots & 1
        \end{bmatrix}
    \end{displaymath}
\end{definition}

\begin{definition}
    \textbf{Diagonal Matrix} \\
    A square matrix in which all off-diagonal entries are \(0\). Diagonal entries may be any scalars.
    \begin{displaymath}
        D
        =
        \begin{bmatrix}
            d_1 & 0 & \cdots & 0 \\
            0 & d_2 & \cdots & 0 \\
            \vdots & \vdots & \ddots & \vdots \\
            0 & 0 & \cdots & d_n
        \end{bmatrix}
    \end{displaymath}
\end{definition}

\begin{definition}
    \textbf{Scalar Matrix} \\
    A diagonal matrix whose diagonal entries are all equal to the same scalar \(\lambda\).
    \begin{displaymath}
        \lambda I_n
        =
        \begin{bmatrix}
            \lambda & 0 & \cdots & 0 \\
            0 & \lambda & \cdots & 0 \\
            \vdots & \vdots & \ddots & \vdots \\
            0 & 0 & \cdots & \lambda
        \end{bmatrix}
    \end{displaymath}
\end{definition}

\begin{definition}
    \textbf{Row Matrix} \\
    A matrix with exactly one row (size \(1 \times n\)).
    \begin{displaymath}
        \begin{bmatrix}
            a_1 & a_2 & \cdots & a_n
        \end{bmatrix}
    \end{displaymath}
\end{definition}

\begin{definition}
    \textbf{Column Matrix} \\
    A matrix with exactly one column (size \(m \times 1\)).
    \begin{displaymath}
        \begin{bmatrix}
            a_1 \\
            a_2 \\
            \vdots \\
            a_m
        \end{bmatrix}
    \end{displaymath}
\end{definition}

\begin{definition}
    \textbf{Square Matrix} \\
    A matrix with the same number of rows and columns (size \(n \times n\)).
    \begin{displaymath}
        A
        =
        \begin{bmatrix}
            a_{11} & a_{12} & \cdots & a_{1n} \\
            a_{21} & a_{22} & \cdots & a_{2n} \\
            \vdots & \vdots & \ddots & \vdots \\
            a_{n1} & a_{n2} & \cdots & a_{nn}
        \end{bmatrix}
    \end{displaymath}
\end{definition}

\newpage
\begin{theorem}
    \textbf{Properties of Matrix Addition} \\
    Let \(A, B,\) and \(C\) be matrices of the same size \(m \times n\). Then:
    \begin{enumerate}
        \item \(A + B = B + A\) \hfill (Commutative law)
        \item \((A + B) + C = A + (B + C)\) \hfill (Associative law)
        \item \(A + A + \cdots + A = kA\) \hfill (\(k\) copies of \(A\))
        \item There exists a zero matrix \(O\) such that \(A + O = A\).
        \item For each \(A\), there exists \(-A\) such that
        \[
            A + (-A) = A - A = O.
        \]
        The matrix \(-A\) is called the additive inverse of \(A\).
    \end{enumerate}
\end{theorem}

\begin{theorem}
    \textbf{Properties of Matrix Multiplication} \\
    Let \(A, B,\) and \(C\) be matrices of appropriate sizes so that each product is defined. Then:
    \begin{enumerate}
        \item \((AB)C = A(BC)\) \hfill (Associative law)
        \item \(A(B + C) = AB + AC\) \hfill (Left distributive law)
        \item \((B + C)A = BA + CA\) \hfill (Right distributive law)
        \item \(AO = OA = O\), where \(O\) is a zero matrix of compatible size.
    \end{enumerate}
\end{theorem}

\begin{theorem}
    \textbf{Scalar-Matrix Distribution Laws} \\
    Let \(A\) and \(B\) be \(m \times n\) matrices, and let \(c, k\) be scalars. Then:
    \begin{enumerate}
        \item \((ck)A = c(kA)\)
        \item \(k(A + B) = kA + kB\)
        \item \((c + k)A = cA + kA\)
    \end{enumerate}
\end{theorem}


% WARNING: STOP!
% ---------------------------------------------------------------------------------------
% WARNING: Do Not Not Edit Below!
% ---------------------------------------------------------------------------------------
% WARNING: STOP!
% ---------------------------------------------------------------------------------------


% Part: Part I: Part Name
% ---------------------------------------------------------------------------------------
\part{Part I: Part Name}

% Chapter: Chapter Name
% ---------------------------------------------------------------------------------------
\chapterimage{chapter-cover-two.jpg} % Chapter heading image
\chapterspaceabove{6.75cm} % Whitespace from the top of the page to the chapter title on chapter pages
\chapterspacebelow{7.25cm} % Amount of vertical whitespace from the top margin to the start of the text on chapter pages
\chapter{Introductions}\index{Introductions}

% Section: Section Name
% ---------------------------------------------------------------------------------------
\section{Welcome}\index{Introductions!Welcome}
This is a simple paragraph in LaTeX.

% Part: Part I: Part Name
% ---------------------------------------------------------------------------------------
\part{Part II: Examples}

% Chapter: Chapter Name
% ---------------------------------------------------------------------------------------
\chapterimage{chapter-cover-two.jpg} % Chapter heading image
\chapterspaceabove{6.75cm} % Whitespace from the top of the page to the chapter title on chapter pages
\chapterspacebelow{7.25cm} % Amount of vertical whitespace from the top margin to the start of the text on chapter pages
\chapter{Introductions}\index{Introductions}

% Section: Introductions
% ---------------------------------------------------------------------------------------
\section{Basics}\index{Introductions!Basics}
This is a simple paragraph in LaTeX.

This is a new paragraph. Create a new paragraph by leaving a blank line.

This line ends here.\\
This line starts after a manual line break.

You can make text \textbf{bold}, \textit{italic}, or \textbf{\textit{both}}. \\
\noindent
You can also use \emph{emphasis}, which is usually italicized.

% Subsection: Lists
% ---------------------------------------------------------------------------------------
\subsection{Enumerated List}\index{Basics!Enumerate}
\begin{enumerate}
    \item First numbered item
        \begin{enumerate}
            \item First indented numbered item
            \item Second indented numbered item
                \begin{enumerate}
                    \item First second-level indented numbered item
                \end{enumerate}
        \end{enumerate}
    \item Second numbered item
    \item Third numbered item
\end{enumerate}

% Subsection: Lists
% ---------------------------------------------------------------------------------------
\subsection{Itemize List}\index{Basics!Itemize}
\begin{itemize}
    \item First bullet point item
        \begin{itemize}
            \item First indented bullet point item
            \item Second indented bullet point item
                \begin{itemize}
                    \item First second-level indented bullet point item
                \end{itemize}
        \end{itemize}
    \item Second bullet point item
    \item Third bullet point item
\end{itemize}

\newpage
% Section: Matricies and Tables
% ---------------------------------------------------------------------------------------
\section{Matrices and Tables}\index{Introductions!Tables}

% Subsection: Matrix
% ---------------------------------------------------------------------------------------
\subsection{Matrix}\index{Tables!Matrix}
\[
    A =
    \begin{bmatrix}
    1 & 2 & 3 \\
    4 & 5 & 6
    \end{bmatrix}
\]

% Subsection: Matrix
% ---------------------------------------------------------------------------------------
\subsection{Augmented Matrix}\index{Tables!Matrix}
\[
    M_{1} =
    \left[
        \begin{array}{ccc|c}
        1 & 0 & -2 & 3 \\
        0 & 1 &  4 & 6 \\
        2 & -1 & 0 & -3
        \end{array}
    \right]
\]

% Subsection: Tables
% ---------------------------------------------------------------------------------------
\subsection{Tables}\index{Tables!Matrix}

\begin{table}[h]
    \centering
    \renewcommand{\arraystretch}{1.25} % spacing
        \begin{tabular}{||c c c c c c||}
            \hline
            $x$     & 0    & 1    & 2    & 3    & 4    \\
            \hline
            $f(x)$  & 0.41 & 0.37 & 0.16 & 0.05 & 0.01 \\
            \hline
        \end{tabular}

    \caption{Example Table I}
    \label{tab:exampletableone}
\end{table}
Referencing \autoref{tab:exampletableone} in-text using its label.

% Section: Figures 
% ---------------------------------------------------------------------------------------
\section{Figures}\index{Introductions!Figures}
\begin{figure}[H] % Use [H] to suppress floating and place the figure/table exactly where it is specified in the text
    \centering % Horizontally center the figure on the page
    \includegraphics[width=0.35\textwidth]{background.pdf} % Include the figure image
    \caption{Example Figure.}
    \label{fig:example} % Unique label used for referencing the figure in-text
\end{figure}
Referencing \autoref{fig:example} in-text using its label. \\

% \begin{figure}[b] % Floating figure, [b] tells LaTeX to place it at the bottom of the next available page
%     \centering % Horizontally center the figure on the page
%     \includegraphics[width=\textwidth]{background.pdf} % Include the figure image
%     \caption{Floating figure.}
%     \label{fig:floating} % Unique label used for referencing the figure in-text
% \end{figure}

\newpage
% Section: Citations 
% ---------------------------------------------------------------------------------------
\section{Citations}\index{Introductions!Citations}
According to Walpole (2019), "students often had difficulty using \textbf{APA style}, especially when it was their first time" (p. 1).

\noindent
This statement requies a \textbf{citation} \cite{rosen-2019} found "students often had difficulty using APA style".

This statement involkes a \textbf{footnote}, Walpole (2019) found "students often had difficulty using APA style".
\footnote{Ronald Walpole, “Scientific Data,” essay, in Introduction to Statistics and Probability
for Engineers and Scientists, ed. Raymond Myers, 9th ed. (Boston, MA: Pearson, 2019), p.1}

% Section: Links
% ---------------------------------------------------------------------------------------
\section{Link}\index{Introductions!Links}

This is a URL link: \href{https://www.latextemplates.com}{LaTeX Templates}. This is an email link: \href{mailto:example@example.com}{example@example.com}. This is a monospaced URL link: \url{https://www.LaTeXTemplates.com}.

% Section: Code
% ---------------------------------------------------------------------------------------
\section{Code}\index{Introductions!Code}
\begin{lstlisting}[language=C]
/* prompt user for input */
printf("prompt: ");

/* scanf standard input stream */
scanf("%modifier", address);
\end{lstlisting}

\lstinputlisting[label=snippet, caption=Coding Snippet, language=C]{source/struct.c}

% Section: Basic Mathematics
% ---------------------------------------------------------------------------------------
\section{Basic Mathematics}\index{Introductions!Mathematics}
Displayed equations appear centered on their own line: \\
Inline math appears within text, like $a^2 + b^2 = c^2$.

% Subsection: Fractions
% ---------------------------------------------------------------------------------------
\subsection{Fractions}\index{Mathematics!Fractions}
\[
\frac{x + 1}{x - 3}
\]

% Subsection: Subscripts & Superscripts 
% ---------------------------------------------------------------------------------------
\subsection{Superscripts \& Subscripts}\index{Mathematics!Exponets}

% Subsubsection: Superscripts 
% ---------------------------------------------------------------------------------------
\subsubsection{Superscripts}\index{Exponets!Superscripts}
$x^2$, $a^{n+1}$, $e^{i\pi} + 1 = 0$.

% Subsubsection: Subscripts
% ---------------------------------------------------------------------------------------
\subsubsection{Subscripts}\index{Exponets!Subscripts}
$x_1$, $a_{n}$, $A_{ij}$.

\newpage
% Section: Propositions 
% ---------------------------------------------------------------------------------------
\section{Propositions}\index{Introductions!Propositions}

\begin{proposition} % Specify a name/title in square brackets, or leave them out for no title
    Let $f,g\in L^2(G)$; if $\forall \varphi\in\mathcal{D}(G)$, $(f,\varphi)_0=(g,\varphi)_0$ then $f = g$. 
\end{proposition}

\begin{proposition}[Proposition name] % Specify a name/title in square brackets, or leave them out for no title
    It has the properties:
    \begin{align}
        & \big| ||\mathbf{x}|| - ||\mathbf{y}|| \big|\leq || \mathbf{x}- \mathbf{y}||\\
        &  ||\sum_{i=1}^n\mathbf{x}_i||\leq \sum_{i=1}^n||\mathbf{x}_i||\quad\text{where $n$ is a finite integer}
    \end{align}
\end{proposition}

% Subsection: Several Equations
%------------------------------------------------
\subsection{Several equations}\index{Propositions!SeveralEquations}

\begin{theorem}[Name of the theorem] % Specify a name/title in square brackets, or leave them out for no title
    In $E=\mathbb{R}^n$ all norms are equivalent. It has the properties:
    \begin{align}
        & \big| ||\mathbf{x}|| - ||\mathbf{y}|| \big|\leq || \mathbf{x}- \mathbf{y}||\\
        &  ||\sum_{i=1}^n\mathbf{x}_i||\leq \sum_{i=1}^n||\mathbf{x}_i||\quad\text{where $n$ is a finite integer}
    \end{align}
\end{theorem}

% Section: Theorems
%------------------------------------------------
\section{Theorems}\index{Introductions!Theorems}

\begin{theorem} % Specify a name/title in square brackets, or leave them out for no title
    A set $\mathcal{D}(G)$ in dense in $L^2(G)$, $|\cdot|_0$. 
\end{theorem}

% Section: Definitions Examples 
%------------------------------------------------
\section{Definitions}\index{Introductions!Definitions}

\begin{definition}[Definition name] % Specify a name/title in square brackets, or leave them out for no title
    Given a vector space $E$, a norm on $E$ is an application, denoted $||\cdot||$, $E$ in $\mathbb{R}^+=[0,+\infty[$ such that:
    \begin{align}
        & ||\mathbf{x}||=0\ \Rightarrow\ \mathbf{x}=\mathbf{0}\\
        & ||\lambda \mathbf{x}||=|\lambda|\cdot ||\mathbf{x}||\\
        & ||\mathbf{x}+\mathbf{y}||\leq ||\mathbf{x}||+||\mathbf{y}||
    \end{align}
\end{definition}

% Section: Examples 
%------------------------------------------------
\section{Examples}\index{Introductions!Examples}

\begin{example} % Specify a name/title in square brackets, or leave them out for no title
    Let $G=\{x\in\mathbb{R}^2:|x|<3\}$ and denoted by: $x^0=(1,1)$:
    \begin{equation}
        f(x)=\left\{\begin{aligned} & \mathrm{e}^{|x|} & & \text{si $|x-x^0|\leq 1/2$}\\
        & 0 & & \text{si $|x-x^0|> 1/2$}\end{aligned}\right.
    \end{equation}
    The function $f$ has bounded support, we can take $A=\{x\in\mathbb{R}^2:|x-x^0|\leq 1/2+\epsilon\}$ for all $\epsilon\in\mathopen{]}0\,;5/2-\sqrt{2}\mathclose{[}$.
\end{example}

% Subsection: Text Examples 
%------------------------------------------------
\section{Text Example}\index{Examples!Text}

\begin{example}[Example name] % Specify a name/title in square brackets, or leave them out for no title
    Aliquam arcu turpis, ultrices sed luctus ac, vehicula id metus. Morbi eu feugiat velit, et tempus augue.
    Donec cursus maximus luctus. Vivamus lobortis eros et massa porta porttitor.
\end{example}

\newpage
% Section: Exercises Examples 
%------------------------------------------------
\section{Exercises}\index{Introductions!Exercises}

\begin{exercise} % Specify a name/title in square brackets, or leave them out for no title
    This is a good place to ask a question to test learning progress or further cement ideas into students' minds.
\end{exercise}

% Section: Equation Problems 
%------------------------------------------------
\section{Problems}\index{Introductions!Problems}

\begin{problem} % Specify a name/title in square brackets, or leave them out for no title
    What is the average airspeed velocity of an unladen swallow?
\end{problem}


% Section: Notations
%------------------------------------------------
\section{Notations}\index{Introductions!Notations}
\begin{notation} % Specify a name/title in square brackets, or leave them out for no title
    Given an open subset $G$ of $\mathbb{R}^n$, the set of functions $\varphi$ are:
    \begin{enumerate}
        \item Bounded support $G$;
        \item Infinitely differentiable;
    \end{enumerate}
    a vector space is denoted by $\mathcal{D}(G)$. 
\end{notation}

% Section: Corollaries 
%------------------------------------------------
\section{Corollaries}\index{Introductions!Corollaries}
\begin{corollary}[Corollary name] % Specify a name/title in square brackets, or leave them out for no title
    The concepts presented here are now in conventional employment in mathematics. Vector spaces are taken over the field $\mathbb{K}=\mathbb{R}$, however, established properties are easily extended to $\mathbb{K}=\mathbb{C}$.
\end{corollary}

% Section: Remarks
%------------------------------------------------
\section{Remarks}\index{Introductions!Remarks}
\begin{remark}
    The concepts presented here are now in conventional employment in mathematics. Vector spaces are taken over the field $\mathbb{K}=\mathbb{R}$, however, established properties are easily extended to $\mathbb{K}=\mathbb{C}$.
\end{remark}

% Section: Vocabulary 
%------------------------------------------------
\section{Vocabulary}\index{Introductions!Vocabulary}

\begin{vocabulary}[Word] % Specify a name/title in square brackets, or leave them out for no title
    Definition of word.
\end{vocabulary}

\newpage
% Section: Calculus 
% ---------------------------------------------------------------------------------------
\section{Calculus}\index{Mathematics!Calculus}

% Subsection: Cases 
% ---------------------------------------------------------------------------------------
\subsection{Cases}\index{Calculus!Cases}
\textbf{Cases:} \\
\begin{flalign*}
    \tilde{x} =
    \begin{cases}
        x_{(n-1)/2} & \text{if n is odd} \\
        \frac{1}{2}(x_{n/2} + x_{n/2+1}) & \text{if n is even} \\
    \end{cases}
\end{flalign*} \\

% Subsection: Limits & Summations 
% ---------------------------------------------------------------------------------------
\subsection{Limits \& Summations}\index{Calculus!Summations}
\noindent
\textbf{Summations:} \\
\begin{flalign*}
    \overline{x} = \sum_{i=1}^{n} \frac{x_i}{n} = \frac{x_1 + x_2 + \dots + x_n}{n}
\end{flalign*} \\

\noindent
\textbf{Limits:} \\
\[
\lim_{x \to 0} \frac{\sin x}{x} = 1
\] \\

% Subsection: Derivatives 
% ---------------------------------------------------------------------------------------
\subsection{Derivatives}\index{Calculus!Derivatives}
\textbf{Derivative}
\[
\frac{d}{dx} x^2 = 2x
\] \\

\noindent
\textbf{Higher-order derivative:}
\[
\frac{d^2}{dx^2} x^3 = 6x
\] \\

% Subsection: Integrals 
% ---------------------------------------------------------------------------------------
\subsection{Integrals}\index{Calculus!Integrals}
\textbf{Indefinite:}
\[
\int x^2 \, dx = \frac{x^3}{3} + C
\] \\

\noindent
\textbf{Definite:}
\[
\int_{0}^{1} x^2 \, dx = \frac{1}{3}
\]

% Part: Part: End Matter
% ---------------------------------------------------------------------------------------
\part{Part: End Matter}

%----------------------------------------------------------------------------------------
\stopcontents[part] % Manually stop the 'part' table of contents here so the previous

%----------------------------------------------------------------------------------------
%	INDEX
%----------------------------------------------------------------------------------------

\cleardoublepage % Make sure the index starts on an odd (right side) page
\phantomsection
\addcontentsline{toc}{chapter}{\textcolor{ocre}{Index}} % Add an Index heading to the table of contents
\printindex % Output the index

%----------------------------------------------------------------------------------------
%	BIBLIOGRAPHY
%----------------------------------------------------------------------------------------

% \bibliographystyle{unsrt}
\bibliographystyle{apalike}
\bibliography{references}
\markboth{\sffamily\normalsize\bfseries Bibliography}{\sffamily\normalsize\bfseries Bibliography} % Set the page headers to display a Bibliography chapter name
\addcontentsline{toc}{chapter}{\textcolor{ocre}{Bibliography}} % Add a Bibliography heading to the table of contents

%----------------------------------------------------------------------------------------
%	APPENDICES
%----------------------------------------------------------------------------------------

\chapterimage{chapter-cover-two.jpg} % Chapter heading image
\chapterspaceabove{6.75cm} % Whitespace from the top of the page to the chapter title on chapter pages
\chapterspacebelow{7.25cm} % Amount of vertical whitespace from the top margin to the start of the text on chapter pages

\begin{appendices}
    \renewcommand{\chaptername}{Appendix} % Change the chapter name to Appendix, i.e. "Appendix A: Title", instead of "Chapter A: Title" in the headers
    \chapter{Appendix}
    \section{Appendix Section Title}
    Lorem ipsum dolor sit amet, consectetur adipiscing elit. Aliquam auctor mi risus, quis tempor libero hendrerit at.
    Duis hendrerit placerat quam et semper. Nam ultricies metus vehicula arcu viverra, vel ullamcorper justo elementum.
    Pellentesque vel mi ac lectus cursus posuere et nec ex. Fusce quis mauris egestas lacus commodo venenatis.
    Ut at arcu lectus. Donec et urna nunc. Morbi eu nisl cursus sapien eleifend tincidunt quis quis est. Donec ut orci ex.
    Praesent ligula enim, ullamcorper non lorem a, ultrices volutpat dolor. Nullam at imperdiet urna. 
    Pellentesque nec velit eget est euismod pretium.
\end{appendices}

%----------------------------------------------------------------------------------------
\end{document}
%----------------------------------------------------------------------------------------
